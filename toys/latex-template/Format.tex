%&program=pdflatex
%%%%%%%%%%%%%%%%%%%%%%%%%%%%%%%%%%%%%%%%%%%%%%%%%%%%%%%%%%%%%%%%%%%%%%%%%%%%%%%%
% HUW'S LATEX TEMPLATE
% 
% This is Huw's LaTeX template.
% 
% It is intended to allow me to remove the bits I don't like and keep the 
% formatting and stuff. There's an example of many things here, like maths and
% tables.
% 
% KNOWN ISSUES
% * does not give picture examples
% * does not give reference examples
% * does not give bibliographic examples
%
% SEE ALSO
% * the letter template, in a different place and repository for some reason.
% * the LaTeX wikibook: http://en.wikibooks.org/wiki/LaTeX
%%%%%%%%%%%%%%%%%%%%%%%%%%%%%%%%%%%%%%%%%%%%%%%%%%%%%%%%%%%%%%%%%%%%%%%%%%%%%%%%

\documentclass[11pt]{article}
\usepackage{geometry}
\geometry{a4paper}
%\geometry{landscape}
%\geometry{twocolumn}


% For more Maths symbols
\usepackage{amsmath}
\usepackage{amssymb}

% For importing sweet eps
\usepackage{epstopdf}

% For sweet pseudo code
\usepackage{algorithmic}
\usepackage{algorithm}

% Sweet hyper references.
\usepackage{hyperref}

%%%%%%%%%%%%%%%%%%%%%%%%%%%%%%%%%%%%%%%%%%%%%%%%%%%%%%%%%%%%%%%%%%%%%%%%%%%%%%%%
%%%%%%%%%%%%%%%%%% STYLE OPTIONS %%%%%%%%%%%%%%%%%%%%%%%%%%%%%%%%%%%%%%%%%%%%%%%
%%%%%%%%%%%%%%%%%%%%%%%%%%%%%%%%%%%%%%%%%%%%%%%%%%%%%%%%%%%%%%%%%%%%%%%%%%%%%%%%

% Change fonts
% Palatino for rm | Helvetica for ss | Courier for tt
\renewcommand{\rmdefault}{ppl} % rm
\linespread{1.10}        % Palatino needs more leading
\usepackage[scaled]{helvet} % ss
%\usepackage{courier} % tt
\normalfont
\usepackage[T1]{fontenc}


%Make headings Helvetica
\usepackage{sectsty}
\allsectionsfont{\sffamily}

%Set headers and footers.
%\usepackage{lastpage}
%\usepackage{fancyhdr}
%\pagestyle{fancy} % options: empty , plain , fancy
%\renewcommand{\headrulewidth}{0pt}
%\fancyhead[L]{\sffamily Huw Rowlands}
%\fancyhead[C]{\sffamily Curriculum Vitae}
%\fancyhead[R]{\sffamily \thismonth}
%\fancyfoot[L]{}
%\fancyfoot[C]{}
%\fancyfoot[R]{\sffamily \thepage\ of \pageref{LastPage}}
%
%%Redefining plain style.
%\fancypagestyle{plain}{ 
%	\fancyhf{} % clear all header and footer fields
%	\fancyfoot[R]{\sffamily \thepage\ of \pageref{LastPage}} %except the R footer 
%} 

%%%%%%%%%%%%%%%%%%%%%%%%%%%%%%%%%%%%%%%%%%%%%%%%%%%%%%%%%%%%%%%%%%%%%%%%%%%%%%%%
%%%%%%%%%%%%%%%%%% END STYLE OPTIONS %%%%%%%%%%%%%%%%%%%%%%%%%%%%%%%%%%%%%%%%%%%
%%%%%%%%%%%%%%%%%%%%%%%%%%%%%%%%%%%%%%%%%%%%%%%%%%%%%%%%%%%%%%%%%%%%%%%%%%%%%%%%

%\myname command
\newcommand{\myname} {Huw Rowlands}

%\thismonth shows this month.
\newcommand{\thismonth}{
	\ifcase\month\or
  		January\or February\or March\or April\or May\or June\or
  		July\or August\or September\or October\or November\or December
	\fi
	\space\number\year}

%Make date format of \today command Australian.
\renewcommand{\today}{\number\day\space
	\ifcase\month\or
  		January\or February\or March\or April\or May\or June\or
  		July\or August\or September\or October\or November\or December
	\fi\space
	\number\year}

	
%oldtoday is just like \today, but with old style numbers and small cap months
\newcommand{\oldtoday}{\oldstylenums{\number\day}\space
	\textsc{
	\ifcase\month\or
  		January\or February\or March\or April\or May\or June\or
  		July\or August\or September\or October\or November\or December
	\fi\space}
	\oldstylenums{\number\year}}



%Set up meta data for document.
\hypersetup{
    bookmarks=true,			% show bookmarks bar?
    unicode=true,			% non-Latin characters in Acrobat’s bookmarks
    pdftoolbar=true,			% show Acrobat’s toolbar?
    pdfmenubar=true,			% show Acrobat’s menu?
    pdffitwindow=false,		% window fit to page when opened
    pdfstartview={FitH},		% fits the width of the page to the window
    pdftitle={LaTeX Template},		% title
    pdfauthor={\myname},		% author
    pdfsubject={},			% subject of the document
    pdfcreator={\myname},	% creator of the document
    pdfproducer={\myname},	% producer of the document
    pdfkeywords={},			% list of keywords
    pdfnewwindow=true,		% links in new window
    colorlinks=false,		% false: boxed links; true: colored links
    linkcolor=red,			% color of internal links
    citecolor=green,			% color of links to bibliography
    filecolor=magenta,		% color of file links
    urlcolor=cyan,			% color of external links
    linkbordercolor={1 1 1},	% color of internal links' borders
    citebordercolor={1 1 1},	% color of links to biliography's borders
    filebordercolor={1 1 1},	% color of file links' borders
    urlbordercolor={0 1 1},	% color of external links' borders
}


%%% TITLE 
\title{\LaTeX\ Template}%Don't forget to change the title in hypersetup section!
\author{\href{mailto:huw.rowlands@ieee.org}{\myname}}
\date{\oldtoday} % edit month above.

%%% BEGIN DOCUMENT
\begin{document}

%%% MAKE TITLE
\maketitle
\tableofcontents


%%% ABSTRACT
\begin {abstract}
The rest of this document will not make sense and it is not intended to make sense. It is intended to have most of its paragraphs removed and replaced with stuff that does make sense, but formatted nicely.

Copy with:

\begin{verbatim}
cp /Users/huwr/Documents/LaTeX\ template/format.tex ~/Desktop/
\end{verbatim}

The most useful website you can have open when typesetting is the \href{http://en.wikibooks.org/wiki/LaTeX}{\LaTeX\ wiki-book}.


This is under version control with \texttt{git}. It was written on 27\ January\ 2010.
\end {abstract}

%%% UNNUMBERED PARAGRAPH
% This bit for the putting it into the TOC.
%\phantomsection
%\addcontentsline{toc}{section}{Unnumber paragraph}
\section*{Unnumbered paragraph}
Mauris molestie bibendum dapibus. Mauris congue ante in odio gravida posuere sit amet et quam. Etiam adipiscing purus nec ligula vehicula in accumsan urna sollicitudin. Pellentesque habitant morbi tristique senectus et netus et malesuada fames ac turpis egestas. Pellentesque ullamcorper, diam quis auctor posuere, nisi lectus ornare nulla, sodales aliquet nulla erat in orci. Pellentesque interdum purus malesuada purus bibendum id imperdiet sapien elementum. Nam velit dolor, tempus et auctor eget, mattis in lectus. Nam vitae neque sollicitudin libero ornare rhoncus. Donec non quam id turpis hendrerit commodo ut porta ante.


%%% PERSONAL DETAILS
\section{Plain paragraph}
Lorem ipsum dolor sit amet, consectetur adipiscing elit. Ut rutrum metus quis ipsum malesuada imperdiet. Donec rutrum elementum consequat. Suspendisse potenti. Sed sed lectus vulputate felis imperdiet lobortis ac sed sapien. Pellentesque suscipit quam ac nisi laoreet hendrerit sit amet et sapien. Mauris vitae venenatis nisi. Nulla facilisi. Etiam tempor tincidunt enim in ultricies. Vestibulum scelerisque molestie tellus, a convallis nunc auctor ac. Cras tempor scelerisque ligula, nec gravida eros tincidunt quis. Donec sit amet purus purus, ac facilisis ante. Suspendisse mollis pretium urna, nec pretium lorem tincidunt et.

Suspendisse potenti. Aliquam adipiscing suscipit ligula, in feugiat nulla dignissim eu. Aenean nec dui eu sapien condimentum mattis. Mauris molestie bibendum dapibus. Mauris congue ante in odio gravida posuere sit amet et quam. Etiam adipiscing purus nec ligula vehicula in accumsan urna sollicitudin. Pellentesque habitant morbi tristique senectus et netus et malesuada fames ac turpis egestas. Pellentesque ullamcorper, diam quis auctor posuere, nisi lectus ornare nulla, sodales aliquet nulla erat in orci. Pellentesque interdum purus malesuada purus bibendum id imperdiet sapien elementum. Nam velit dolor, tempus et auctor eget, mattis in lectus. Nam vitae neque sollicitudin libero ornare rhoncus. Donec non quam id turpis hendrerit commodo ut porta ante. Curabitur et mauris tellus, nec porta dolor. Ut eu augue urna, ut fermentum libero. Nullam nisi purus, posuere et varius ut, varius vehicula lacus. Proin dignissim odio nec odio porta ac sagittis turpis eleifend. Duis mi lacus, imperdiet et pellentesque ac, laoreet in nisi. Etiam aliquet ornare diam ac lacinia. Pellentesque eu augue et dui vestibulum mollis vel non lectus. Vivamus luctus dui eget massa lacinia nec fermentum dolor tincidunt.

Aliquam nulla diam, elementum et eleifend vel, tincidunt eget neque. Ut ac urna arcu. Phasellus posuere velit et nisl tempor bibendum. Morbi non nulla in augue congue aliquet. Donec tincidunt lobortis massa, ut convallis felis euismod elementum. In hac habitasse platea dictumst. Quisque tincidunt purus sed diam semper sagittis. Proin sed adipiscing ipsum. Sed tristique gravida velit, ac volutpat mi venenatis sit amet. Pellentesque tempus lacus sed purus vestibulum congue. Fusce vitae mauris est, nec sollicitudin justo. Vivamus ac auctor quam. Nam pulvinar tempor laoreet. Etiam malesuada, elit id facilisis interdum, eros urna congue ante, eu auctor justo nulla in odio. Morbi aliquam, nibh quis mattis pellentesque, augue justo ultrices odio, sed semper justo risus eu urna. Nullam rhoncus turpis at enim vulputate sed sagittis odio facilisis.


%%% SUBSECTIOEND PARAGRAPH
\section{Subsectioned Paragraph}
\subsection{Sed tristique gravida velit} %Sed tristique gravida velit
Aliquam nulla diam, elementum et eleifend vel, tincidunt eget neque. Ut ac urna arcu. Phasellus posuere velit et nisl tempor bibendum. Morbi non nulla in augue congue aliquet. Donec tincidunt lobortis massa, ut convallis felis euismod elementum. In hac habitasse platea dictumst. Quisque tincidunt purus sed diam semper sagittis. Proin sed adipiscing ipsum. Sed tristique gravida velit, ac volutpat mi venenatis sit amet. Pellentesque tempus lacus sed purus vestibulum congue. Fusce vitae mauris est, nec sollicitudin justo. Vivamus ac auctor quam. Nam pulvinar tempor laoreet. Etiam malesuada, elit id facilisis interdum, eros urna congue ante, eu auctor justo nulla in odio. Morbi aliquam, nibh quis mattis pellentesque, augue justo ultrices odio, sed semper justo risus eu urna. Nullam rhoncus turpis at enim vulputate sed sagittis odio facilisis.

\subsection{Integer imperdiet fringilla ligula id viverra} %Integer imperdiet fringilla ligula id viverra
Vestibulum neque mauris, pellentesque eget egestas non, ultricies non eros. Integer volutpat, erat placerat dapibus malesuada, sem tortor euismod sem, non eleifend risus erat a ante. Mauris non orci a justo scelerisque pretium. Integer imperdiet fringilla ligula id viverra. Quisque blandit orci vitae dolor hendrerit vestibulum. Praesent porttitor tortor ut turpis volutpat elementum. Nullam sit amet dolor purus. In ultrices placerat metus, sit amet mollis sapien tincidunt vitae. Sed quam urna, rhoncus et ultricies quis, fermentum sed mi. Cras eget mauris sit amet mauris ultricies auctor eu sit amet elit. Ut arcu massa, cursus sed pellentesque id, dignissim semper urna. Nullam fermentum enim sed dolor convallis egestas condimentum nisi ultrices. Etiam magna justo, dignissim vel posuere in, sagittis et odio. Nullam accumsan mi vel nibh rhoncus ac aliquam lorem cursus. Etiam leo nisl, eleifend eu vestibulum sit amet, cursus eget ante. Maecenas ac iaculis nisi. Suspendisse orci felis, luctus sed porttitor quis, convallis et nunc. Fusce semper neque nec augue mollis non ullamcorper risus ultricies.

Suspendisse ac purus non elit sollicitudin iaculis. Integer ut lectus purus, ac iaculis elit. Praesent tellus nisi, condimentum sollicitudin egestas eget, sodales nec felis. Ut sed mi id enim rutrum consequat. Donec consectetur orci eu quam vulputate vel vestibulum mi pretium. Aliquam quis lacus et tortor rhoncus mattis non sit amet felis. Etiam nunc dui, posuere eu porta et, varius a erat. Integer dolor massa, sagittis sit amet vehicula ac, consectetur non diam. Mauris tristique odio et quam sollicitudin venenatis. Suspendisse cursus vestibulum massa, a ultrices massa fermentum vitae. Vivamus sed magna diam, sit amet elementum magna.

%%% CLIP
\section{Clip}
Everything past about here is probably not going to be useful for you, but it's here in case.

%%% BASIC TABLE
\section{A basic table}
\begin{tabular}{p{3.5cm} l}
Institution:		  &Lake Tuggeranong College				\\
%Period:			  &30 January 2004 to 16 December 2005	\\
Focus:			  &IT, Science and Mathematics			\\
UAI:				  &91.80									\\
Awards:			  &Achieved highest score for IT			\\
\end{tabular}


%%% ITEMISED LIST
\section{An itemised/itemized list}
\begin{itemize}
\item Cows
\item Sheep
\item Dogs
\item Chickens
\end{itemize}


%%% ENUMERATED LIST
\section{Enumerated list}
\begin{enumerate}
\item Computer systems
\item User interfaces
\item History of computing
\item The Welsh language
\end{enumerate}




%%% MATHEMATICAL EQUIATONS
\section{Basic mathematical equations}
An isomeric transition is where a metastable isotope releases energy in the form of a photon. Metastable means  the ability of an non-equilibrium state to remain for an extended period of time. This means that some isotopes that are metastable remain in an excited state for a period of time after a cause of excitement has occurred, such as a nuclear reaction. As the isotope calms down energy is released, typically in the form of photons.

On such example of a metastable isotope is as such:

\begin{displaymath}
^{137}_{55}Cs \longrightarrow ^{137}_{56}Ba^{\ast} + ^{0}_{-1}e \longrightarrow ^{137}_{56}Ba + ^{0}_{-1}e + \gamma
\end{displaymath}

The period of time it takes for the half of an amount of the isotope to calm down is the half-life of the isotope.

Given that the radiation spreads out from the source evenly in all directions, it would be reasonable to assume that the intensity of the radiation ($R$) at a certain distance ($d$) would be inversely related to the square of the distance:

\begin{displaymath}
	R \propto \frac{1}{d^{2}} = d^{-2}
\end{displaymath}

We have decided to speak in the first person plural when we discourse mathematics. The conjecture is true, since:
\begin{eqnarray}
	f(n) &\leq& f(n) + g(n) \\
	g(n) &\leq& g(n) + f(n)
\end{eqnarray}

Combining $(1)$ and $(2)$ we get:

\begin{eqnarray*}
	min\{f(n), g(n)\} &\leq& f(n) + g(n) \\
    min\{huw(n)\} &\approx& wow(n) + cool(n)
\end{eqnarray*}



	\begin{equation*}
		\sum_{k=1}^{n} \frac{k^{3}}{2^{k}} = \Theta(1)
	\end{equation*}

Let's say the sequence is $P_{i} = p_1, \ldots, p_i$. The length of the longest rising trend of $P_i$ is $L(i)$. If $p_i$ is greater than the biggest element in the rising trend $P_{i-1}$ then $L(i) = L(i-1) + 1$ else $L(i) = L(i-1)$.

\begin{eqnarray*}
L(0) &=& 0 \\
L(x) &=& max_{1 \leq j < i}\{L(j) + 1 | p_{j} < p_{i}\}
\end{eqnarray*}


%%% CODE
\section{Code}
\subsection{Verbatim code} %Verbatim code - code that is formatted verbatim without any changes or highlights.

This is not any particular language. This is pseudo-code.

\begin{verbatim}

/* p is the set of input data */
funk(p)
   n = length(p);
   
   next[n]; /* the next one in the sequence */
   size[n]; /* the size of the best path so far */
   
   max_size; /* the maximum size found */
   max_node; /* the end of a max sequence */
   
   for i = 1 to n do
      next[i] = -1; /* assume no legal move from here */
      size[i] = 1;
      
      for j = 1 to i - 1 do
         if (p[j] < p[i]) && (size[i] < size[j] + 1) then
            size[i] = size[j] + 1;
            next[i] = j;
            
      if (size[i] > max_size) then
         max_size = size[i];
         max_node = i;
         
   result[n]; /* a linked list - the rising trend */
   next_node; /* the next node in the trend */
   while (next_node > -1)
      result.add(p[next_node]);
      next_node = next[next_node];
      
   reverse(result); /* the results would be backwards otherwise */
   print(result);
\end{verbatim}


\subsection{Algorithm code} %Using the algorithm package. Code that is highlighted and mathematical.

\begin{algorithm}                     
\caption{Calculate $y = x^n$}         
\label{alg1}                          
\begin{algorithmic}                   
\REQUIRE $n \geq 0 \vee x \neq 0$
\ENSURE $y = x^n$
\STATE $y \Leftarrow 1$
\IF{$n < 0$}
\STATE $X \Leftarrow 1 / x$
\STATE $N \Leftarrow -n$
\ELSE
\STATE $X \Leftarrow x$
\STATE $N \Leftarrow n$
\ENDIF
\WHILE{$N \neq 0$}
\IF{$N$ is even}
\STATE $X \Leftarrow X \times X$
\STATE $N \Leftarrow N / 2$
\ELSE[$N$ is odd]
\STATE $y \Leftarrow y \times X$
\STATE $N \Leftarrow N - 1$
\ENDIF
\ENDWHILE
\end{algorithmic}
\end{algorithm}

Also try using the \texttt{listings} package.


\appendix
\section{The Senate}
The Senate shall be composed of senators for each State, directly chosen by the people of the State, voting, until the Parliament otherwise provides, as one electorate.

But until the Parliament of the Commonwealth otherwise provides, the Parliament of the State of Queensland, if that State be an Original State, may make laws dividing the State into divisions and determining the number of senators to be chosen for each division, and in the absence of such provision the State shall be one electorate.

Until the Parliament otherwise provides there shall be six senators for each Original State. The Parliament may make laws increasing or diminishing the number of senators for each State, but so that equal representation of the several Original States shall be maintained and that no Original State shall have less than six senators. The senators shall be chosen for a term of six years, and the names of the senators chosen for each State shall be certified by the Governor to the Governor-General.

\end{document} 
